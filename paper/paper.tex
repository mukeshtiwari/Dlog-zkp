%%
%% This is file `sample-sigplan.tex',
%% generated with the docstrip utility.
%%
%% The original source files were:
%%
%% samples.dtx  (with options: `all,proceedings,bibtex,sigplan')
%% 
%% IMPORTANT NOTICE:
%% 
%% For the copyright see the source file.
%% 
%% Any modified versions of this file must be renamed
%% with new filenames distinct from sample-sigplan.tex.
%% 
%% For distribution of the original source see the terms
%% for copying and modification in the file samples.dtx.
%% 
%% This generated file may be distributed as long as the
%% original source files, as listed above, are part of the
%% same distribution. (The sources need not necessarily be
%% in the same archive or directory.)
%%
%%
%% Commands for TeXCount
%TC:macro \cite [option:text,text]
%TC:macro \citep [option:text,text]
%TC:macro \citet [option:text,text]
%TC:envir table 0 1
%TC:envir table* 0 1
%TC:envir tabular [ignore] word
%TC:envir displaymath 0 word
%TC:envir math 0 word
%TC:envir comment 0 0
%%
%%
%% The first command in your LaTeX source must be the \documentclass
%% command.
%%
%% For submission and review of your manuscript please change the
%% command to \documentclass[manuscript, screen, review]{acmart}.
%%
%% When submitting camera ready or to TAPS, please change the command
%% to \documentclass[sigconf]{acmart} or whichever template is required
%% for your publication.
%%
%%
\documentclass[sigconf]{acmart}

%%
%% \BibTeX command to typeset BibTeX logo in the docs
\AtBeginDocument{%
  \providecommand\BibTeX{{%
    Bib\TeX}}}

%% Rights management information.  This information is sent to you
%% when you complete the rights form.  These commands have SAMPLE
%% values in them; it is your responsibility as an author to replace
%% the commands and values with those provided to you when you
%% complete the rights form.
\setcopyright{acmlicensed}
\copyrightyear{2018}
\acmYear{2018}
\acmDOI{XXXXXXX.XXXXXXX}

%% These commands are for a PROCEEDINGS abstract or paper.
\acmConference[Conference acronym 'XX]{Make sure to enter the correct
  conference title from your rights confirmation emai}{June 03--05,
  2018}{Woodstock, NY}
%%
%%  Uncomment \acmBooktitle if the title of the proceedings is different
%%  from ``Proceedings of ...''!
%%
%%\acmBooktitle{Woodstock '18: ACM Symposium on Neural Gaze Detection,
%%  June 03--05, 2018, Woodstock, NY}
\acmISBN{978-1-4503-XXXX-X/18/06}


%%
%% Submission ID.
%% Use this when submitting an article to a sponsored event. You'll
%% receive a unique submission ID from the organizers
%% of the event, and this ID should be used as the parameter to this command.
%%\acmSubmissionID{123-A56-BU3}

%%
%% For managing citations, it is recommended to use bibliography
%% files in BibTeX format.
%%
%% You can then either use BibTeX with the ACM-Reference-Format style,
%% or BibLaTeX with the acmnumeric or acmauthoryear sytles, that include
%% support for advanced citation of software artefact from the
%% biblatex-software package, also separately available on CTAN.
%%
%% Look at the sample-*-biblatex.tex files for templates showcasing
%% the biblatex styles.
%%

%%
%% The majority of ACM publications use numbered citations and
%% references.  The command \citestyle{authoryear} switches to the
%% "author year" style.
%%
%% If you are preparing content for an event
%% sponsored by ACM SIGGRAPH, you must use the "author year" style of
%% citations and references.
%% Uncommenting
%% the next command will enable that style.
%%\citestyle{acmauthoryear}


%%
%% end of the preamble, start of the body of the document source.
\begin{document}

%%
%% The "title" command has an optional parameter,
%% allowing the author to define a "short title" to be used in page headers.
\title{The Name of the Title Is Hope}

%%
%% The "author" command and its associated commands are used to define
%% the authors and their affiliations.
%% Of note is the shared affiliation of the first two authors, and the
%% "authornote" and "authornotemark" commands
%% used to denote shared contribution to the research.

\author{Mukesh Tiwari}
\affiliation{%
 \institution{Rajiv Gandhi University}
 \streetaddress{Rono-Hills}
 \city{Doimukh}
 \state{Arunachal Pradesh}
 \country{India}}

\author{Berry Schonemark}
\affiliation{%
  \institution{Tsinghua University}
  \streetaddress{30 Shuangqing Rd}
  \city{Haidian Qu}
  \state{Beijing Shi}
  \country{China}}



\author{Bas Spitters}
\affiliation{%
  \institution{The Kumquat Consortium}
  \city{New York}
  \country{USA}}
\email{jpkumquat@consortium.net}

%%
%% By default, the full list of authors will be used in the page
%% headers. Often, this list is too long, and will overlap
%% other information printed in the page headers. This command allows
%% the author to define a more concise list
%% of authors' names for this purpose.
\renewcommand{\shortauthors}{Tiwari et al.}

%%
%% The abstract is a short summary of the work to be presented in the
%% article.
\begin{abstract}
  Abstract goes here
\end{abstract}


\begin{CCSXML}
<ccs2012>
  <concept>
      <concept_id>10002978.10002986.10002990</concept_id>
      <concept_desc>Security and privacy~Logic and verification</concept_desc>
      <concept_significance>500</concept_significance>
      </concept>
</ccs2012>
\end{CCSXML}
  
\ccsdesc[500]{Security and privacy~Logic and verification}

%%
%% Keywords. The author(s) should pick words that accurately describe
%% the work being presented. Separate the keywords with commas.
\keywords{Sigma Protocol, Formal Verification, Coq, Cryptography, Zero-Knowledge Proof}



%%
%% This command processes the author and affiliation and title
%% information and builds the first part of the formatted document.
\maketitle

\section{Introduction}
\begin{itemize}
  \item Explain zero-knowledge proofs (completeness, soundness, zero-knowledge)
  \item Explain sigma protocols (completeness, special-soundness, honest-verifier zero-knowledg) 
  and mention about the Schnorr protocol
  \item Our contribution
  \item Explain the Schnorr protocol
  \item Explain Parallel, And, Eq, Or, and NEQ relations (also explain that 
  most of the formalisation has worked on And and Or relations)
\end{itemize}
Zero-knowledge proofs are a class of cryptographic protocols that allow a
prover to convince a verifier that a statement is true without revealing 
any information about the statement itself. 
Zero-knowledge proofs are possible for all problems in the $NP$ complexity class. 
$NP$ includes problems where a proof of membership can be efficiently 
verified in polynomial time. In NP, a witness (denoted as $w$) is a 
polynomial-length piece of evidence that allows quick
verification (in polynomial time) that a given statement $s$ 
belongs to the language. This verification process can be represented 
using a binary relation $R$ between the set of statements (S) and the 
set of witnesses ($W$), where $(s, w) \in R$ if the witness $w$ 
demonstrates that the statement $s$ is in the language. 

Sigma protocols were first defined by Ronald Cramer [18],
they are a particularly simple and efficient kind of zero-
knowledge proof and have seen wide deployment; they re-
main a leading kind of proof both in terms of simplicity and
deployment but recent advances in succinct zero-knowledge
proofs [26] offer greater efficiency. The first efficient sigma
protocol was introduced by Schnorr in [38], several years
before the class was defined.

Sigma protocols are a class of cryptographic protocols that are used to prove the 
knowledge of a witness for a given statement. 

They are used in various cryptographic 
applications such as identification, authentication, and zero-knowledge proofs. 
The security of these protocols is based on the hardness of certain mathematical 
problems such as the discrete logarithm problem. Amongst sigma protocols, 
one of the most well-known protocols is the Schnorr protocol. In this 
paper, we present a formal verification of the Schnorr protocol in the
Coq proof assistant. 

In this paper, we present a 
formal verification of a sigma protocol in the Coq proof assistant. We use 
the Fiat-Crypto library to implement the protocol and prove its security 
properties. Our work is based on the work of \cite{fiat-crypto} and \cite{pok}.

\section{Sigma Protocols}

  Explain here the definition of Sigma Protocols and the Schnorr protocol. 

  \subsection{Parallel Composition}
  Explain the parallel composition of sigma protocols.

  \subsection{And Composition}
  Explain the and composition of sigma protocols.

  \subsection{Or Composition}
  Explain the or composition of sigma protocols.

  \subsection{Equality Composition}
  Explain the equality composition of sigma protocols.

  \subsection{Inequality Composition}
  Explain the inequality composition of sigma protocols.

\section{Formal Verification of Sigma Protocols}


\section{Web Assembly and Rust from the Coq Formalisation}
  Bas Spitter


\section{Case Studies}
  We use our formalisation to model the Cryptographic protocols implemented in 
  Helios and Belenois voting systems. 


\section{Related Work}
  There are three work. Giles Barthe does not extract to executable code, 
  CryptoHol uses law of excluded middle, and thomas haines work is more about 
  verifying a transcript rather than constructing it. Moreover, 
  it does not reason about probabilities in intuitive way. 

\section{Conclusion, Future Work, and Limitations}





%%
%% The next two lines define the bibliography style to be used, and
%% the bibliography file.
\bibliographystyle{ACM-Reference-Format}
\bibliography{reference}


%%
%% If your work has an appendix, this is the place to put it.
%\appendix

\end{document}
\endinput
%%
%% End of file `sample-sigplan.tex'.
